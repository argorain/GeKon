\documentclass[a4paper,11pt]{scrartcl}
\usepackage[T1]{fontenc}
\usepackage[utf8]{inputenc}
%\usepackage{lmodern}
\usepackage[top=1in, bottom=1in, left=0.75in, right=0.75in]{geometry}
\usepackage[czech]{babel}

\usepackage{minted}
\newminted{ocaml}{fontsize=\footnotesize}
\newminted{cpp}{fontsize=\footnotesize}

\title{GeKon}
\subtitle{Projekt MAPV}
\author{Vojtěch Vladyka a Martin Sehnoutka}

\newcommand{\keyword}{\textbf }

\begin{document}

\maketitle

\section{Zadání}
Cílem projektu je navrhnout algoritmus pro indukci konvoluční masky na základě dvou šedotónových nebo barevných obrázků. Pro hledání řešení jsou použité genetické algoritmy. Jako prostředek realizace je zvolena knihovna OpenCV a programovací jazyk C++.

\section{Teoretický rozbor}

Práci je možné rozdělit do dvou základních oblastí. První jsou genetické algoritmy, které byly použity pro hledání optimálního řešení. Druhá oblast je zpracování obrazu, jelikož se zabýváme porovnáváním dvou obrázků, námi vytvořeného a zadaného.

\subsection{Genetické algoritmy}

operátory

\subsection{Zpracování obrazu}

IQA ...

Příklad vysázení kódu pomocí minted:
\begin{cppcode}
template<typename T>
T sum_vec(std::vector<T> vec) {
    T ret = 0;
    std::for_each(vec.begin(), vec.end(), [&](auto it) { ret += it; });
    return ret;
}
\end{cppcode}

\end{document}
